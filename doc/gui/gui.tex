\chapterauthor{Ricardo Krause}
\graphicspath{{./gui/Bilder/}}

\chapter{Arbeitspaket GUI}
\label{GUI}
Für die Analyse, den Entwurf und die Implementierung dieses Arbeitspaketes übernahm Herr Krause die Verantwortung. Er wurde bei der Implementierung von den selbst erstellten Steuerelementen vom Herrn Krupinski unterstützt. Während der Integration des Arbeitspaketes waren die Herren Krupinski und Schleinkofer unterstützend beteiligt.
\paragraph{}
In der Projektspezifikation wurde die Anforderung der Visualisierung von Motor-Charakteristiken und die Steuerung des Motors mit aufgenommen. Zu diesem Zweck wurde eine Benutzeroberfläche in der Programmiersprache C\# mit Hilfe des Grafik-Framework\footnote{Rahmenstruktur in der Software} \textit{WPF} \footnote{Windows Presentation Foundation} von Microsoft erstellt.

\section{Analysephase}
Das Arbeitspaket GUI wurde mit einer Analysephase begonnen. In dieser Phase wurden die möglichen funktionalen und nicht-funktionalen Anforderungen in einer Spezifikation zusammengetragen und aufgelistet. Es wurden Stakeholder\footnote{Projektbeteiligte, welche nicht an der Entwicklung beteiligt sind} befragt, um die gewünschten Anforderungen an das System zu erhalten und anschließend wurden die Ergebnisse der Befragung validiert, priorisiert und in die Spezifikation mit aufgenommen. Die umzusetzenden Projektinhalte wurden dann aus dieser abgleitet.\\
Die wichtigsten drei Anforderungen jeder Gruppe, basierend auf der Analyse: \\

\begin{minipage}[t]{0.4\textwidth}
\textbf{Funktionale Anforderungen:}
\begin{itemize}
	\item Anzeige von Sensordaten in der GUI
	\item Einstellung möglicher Parameter des Motors
	\item Justierung der PID Regelung in der GUI
	\end{itemize} 
\end{minipage}\hfill\begin{minipage}[t]{0.5\textwidth}
\textbf{Nicht-Funktionale Anforderungen:}
\begin{itemize}
	\item Die Software soll modular aufgebaut und erweiterbar sein
	\item Es soll das aktuelle Metro-Design verwendet werden
	\item Der Aufbau soll möglichst einfach gehalten sein, um die Stabilität und Benutzbarkeit zu verbessern.
\end{itemize} 
\end{minipage}
\newline
\newline
\newline
Als weiteres Ergebnis der Analyse ergab sich eine zwingende Abhängigkeit zu dem Arbeitspaket Kommunikation.  Erforderliche Schnittstellen und Datenstrukturen wurden zu diesem Zweck in enger Zusammenarbeit erstellt. Diese sind ebenfalls in die Spezifikation mit aufgenommen worden.
\section{Entwurfsphase}
Mit den Ergebnissen der Analysephase, konnte ein Entwurf der Benutzeroberfläche erstellt werden. Dieser wurde auf der einen Seite für die Code Struktur und Architektur durchgeführt, auf der anderen Seite wurden mögliche Designentwürfe des Layouts erstellt.\\

Als grundlegende Architektur wurde ein drei Schichtenmodell \ref{fig:arc} verwendet, welches den Anforderungen der Analysephase gerecht wurde und allen Entwicklern des Projektes bekannt war. In diesem gewählten Modell kommunizieren nur benachbarte Schichten miteinander. Und jede Schicht hat eine separate Aufgabe.

\begin{figure}[ht]
	\centering
		\includegraphics[scale=0.5]{Arc1}
		\caption{Grober Architektur Entwurf}
		\label{fig:arc}
\end{figure}

Es kamen mehrere Entwurfsmuster zum Einsatz, welche zur Softwarequalität beitragen und dem Entwicklern einzelne Lösungsgerüste für wiederkehrende Probleme liefern.\\

Für die oberen beiden Schichten wurde das MVVM\footnote{kurzw. für Model-View-ViewModel} Entwurfsmuster \ref{fig:mvvm} gewählt. Dieses Muster trennt die Oberfläche von der zugrundeliegenden Business-Logik, dies ermöglichte es die Komponenten unabhängig und austauschbar von der Businesslogik zu entwickeln.\\ 

\begin{figure}[ht]
	\centering
		\includegraphics[width=0.6\textwidth]{MVVM}
		\caption{MVVM-Konzept}
		\label{fig:mvvm}
\end{figure}

Weitere eingesetzte Entwurfsmuster sind das Kommandomuster, welches alle ausgeführten Events auf der Benutzeroberfläche auffängt und im ViewModel behandelt und verarbeitet, das Repository\footnote{engl. Haufen}-Muster, welches die Datenverwaltung und Kommunikation zu Subsystemen übernimmt und flexible auf Änderungen im Kommunikationsmodul reagiert. Das "Umkehrung der Kontrolle"-Muster, für dessen Zweck ein Abhängigkeits-Container zu Beginn des Systemstarts erstellt und mit allen bekannten Abhängigkeiten gefüllt wird. Anschließend ist der Zugriff auf die im Container befindlichen Objekte von jedem Punkt der Anwendung her ermöglicht.\\

Wie bereits in der Analysephase erwähnt, wurden die Datenstrukturen dem Kommunikationsmodul angepasst siehe Abbildung \ref{lst:SensorData}.\\
\begin{lstlisting}[frame=single, caption=Beschreibung der Sensordatenstruktur, label=lst:SensorData]
public class SensorData
{
public ulong Timestamp { get; set; }
public SortedList<ushort, double> DataTable { get; set; }
}
\end{lstlisting}
Im zweiten Teil des Entwurfes wurde ein erstes einfaches Layout \ref{fig:gui} erstellt. Für dieses wurde das Werkzeug Pencil\footnote{Link zum Projekt \url{http://pencil.evolus.vn/}} eingesetzt. Es ist einfach zu bedienen und lieferte schnell ein Ergebnis, welches im Projektteam besprochen werden konnte.

\begin{figure}[ht]
	\centering
		\includegraphics[width=0.5\textwidth]{Mockup}
		\caption{Erster Layout Entwurf der GUI}
		\label{fig:gui}
\end{figure}

\section{Implementierung}

In diesem Abschnitt wurden die Ergebnisse der Entwurfsphase umgesetzt und die Benutzeroberfläche erstellt. Das Programm wurde mit dem Visual Studio 2015 in der Programmiersprache C\# und den .NET Framework WPF erstellt. In dieser Phase kommen die vom Herrn Krupinski erstellten Steuerelemente zum Einsatz.\\

Zu Beginn der Implementierung wurde die Frage aufgeworfen, welche Technologie für die Oberfläche zum Einsatz gebracht werden sollten. Zur Auswahl stand, es mit neuesten Multi-Plattform fähigen Web-Technologien umzusetzen und auf das ASP.NET-Framework\footnote{Microsoft Framework zur Entwicklung von Web-Applikationen} zu setzen oder eine reine Desktop Applikation mit dem WPF-Framework umzusetzen. Aufgrund des relativ knappen Zeitplanes für das Projekt und der mehrheitlichen Erfahrungen im WPF-Framework, wurde sich für dieses entschieden.\\

Als Entwicklungsumgebung wurde das Visual Studio Enterprise 2015 eingesetzt. Dieses Tool bietet dem Entwickler einen sehr kraftvollen Code-Editor, einfach zu bedienenden Debugger, umfangreiche Diagnose Tools, integriertes Test Framework und ist flexibel erweiterbar zusätzlich werden Projekt Templates bereitgestellt, welche schon grundlegende Konzepte wie Architektur Muster mitbringen.\\

Als Grundlage für den Start des Projektes wurde das MVVM-Light\footnote{Link zum Projekt\url{https://mvvmlight.codeplex.com/}} Toolkit verwendet. Es beinhaltet grundlegende Funktionalitäten und Templates, welche den Entwickler, indem es Basistechnologien wie die Architektur Struktur MVVM und DI-Techniken mitbringt, entlasten. Dieses Toolkit ist ebenfalls als Erweiterung direkt im Visual Studio verfügbar.\\

Der Datenzugriff wurde durch den Einsatz vom Repository-Muster (siehe Abbildung \ref{fig:repo}) realisiert. Dies ermöglichte zu Beginn des Projektes und für Test gegen eine Attrappe des Kommunikationsmoduls zu entwickeln. Diese täuschte das fehlende Modul vor und konnte problemlos durch das fertig implementierte Modul ersetzt werden. Durch die vorangegangene Schnittstellendefinition mussten für Integration des Kommunikationsmoduls keine Änderungen an der internen Systemlogik vorgenommen werden.   

\begin{figure}[ht]
	\centering
		\includegraphics[width=0.5\textwidth]{RepoPattern}
		\caption{Repository Muster in der GUI}
		\label{fig:repo}
\end{figure}

Ein weiteres Problem, welches es zu lösen gab, waren die Steuerelemente für die Anzeige der Sensorwerte. Es folgte eine kurze Evaluation und Testphase für die möglichen Tachometer- und Linienchart-Anzeigen, leider mit dem Ergebnis, dass die fertigen Produkte entweder Kostenpflicht waren oder schlecht dokumentiert und nicht genügend Anpassbar für unseren Zweck waren.\\

Daraufhin erstellte der Herr Krupinski diese beiden aufwändigen Steuerelemente als benutzerdefinierte WPF-Steuerelemente selbst, welche uns die geforderten Funktionalitäten und die notwendige Veränderbarkeit mitbrachten. 



\subsection*{Gauge Control}
Das entwickelte "Gauge"\footnote{Englisch für Tachometer.} Control in Abbildung \ref{fig:gauge} besteht aus einer Skala(1) mit einem minimal und maximal Wert, einen Zeiger(2) und einem Label(3), welches den Aktuellen Wert anzeigt.

\begin{figure}[ht]
	\centering
		\includegraphics[width=0.3\textwidth]{Gauge1}
		\caption{Tachometer Anzeige der GUI}
		\label{fig:gauge}
\end{figure}

\subsection*{LineChart Control}

In Abbildung \ref{fig:line} ist das "LineChart"\footnote{Englisch für Liniendiagramm.} Control abgebildet. Es besteht aus einem anzeige Raster (1), einen Graphen (2) zur Visualisierung der Werte einer Anzeige für die Y-Achse (3) mit einstellbaren Werten und einer Anzeige für den ersten und letzten Wert der X-Achse (4).

\begin{figure}[ht]
	\centering
	\includegraphics[width=0.6\textwidth]{Line1}
	\caption{Liniendiagramm Anzeige der GUI}
	\label{fig:line}
\end{figure}

\newpage

\subsection{Implementierung Gauge und Line Control}
\chapterauthor{Bernd Krupinski}
\subsubsection{Gauge Control Implementierung}

Die Implementierung der Gauge Control erfolgte in der Windows Presentation Foundation (WPF) über eine Custom Control. Zur Darstellung der Skala(1) und dem Zeiger(2) wurden jeweils ein Canvas verwendet. Ein Canvas stellt in WPF eine Zeichenoberfläche dar.\\



Das Control stellt eine Reihe an Variablen zur Verfügung. Diese ermöglichen es dem Entwickler das Control mit Hilfe von wenigen Zeilen Code stark anzupassen. 
\begin{figure}[ht]
	\centering
	\includegraphics[width=0.4\textwidth]{GaugeDetails}
	\caption{Gauge Control Details}
	\label{fig:gauge1}
\end{figure}
Umfassen dabei sind visuelle Parameter wie Kreishintergrundfarbe(G1), Strichlänge, Strichfarbe, Strichanzahl(G2), innerer Radius(G3), Nadellänge, Nadelfarbe(G6) oder Nadelfüllfarbe(G5). Zusätzlich dazu funktionelle Parameter wie Maximalwert, Minimalwert, aktueller Wert etc.\\\\

Das Canvas Control benutzt den Render Zyklus von WPF. Das heißt dass von unserer Seite das Canvas nur geändert werden muss, sollte einer der genannten Variablen sich verändern, also ein neu zeichnen des Tachometers notwendig ist.\\
Dies geschieht in 2 Schritten. \\
\begin{itemize}
	\item Zeichnen des Hintergrunds (Canvas 1)
	\item Zeichnen der Nadel (Canvas 2)
\end{itemize}

Der Hintergrund besteht wiederum aus 2 Teilen. \\

\begin{itemize}
	\item Der kreisförmige Hintergrund.
	\item Halbkreis aus Strichen.
\end{itemize}
Während die Striche mit einer Menge an Strich-Formen entlang des inneren Radius mit der Länge Äußerer Radius(G4) - Innerer Radius, einfach gelöst wird, stellt der Hintergrund eine etwas interessante Problematik.\\
Zwar gibt es eine Ellipsen Form für WPF's Canvas Control, allerdings besteht die Anforderung aus keiner rein einfarbigen Ellipse, sondern stattdessen aus einer zwei geteilte Form wie unten abgebildet.
\begin{figure}[ht]
	\centering
	\includegraphics[width=.4\textwidth]{GaugeForm}
	\caption{Gauge Control Details}
	\label{fig:gauge2}
\end{figure}
Stattdessen wird ein Polygon benutzt. Ein Polygon ist eine mit einer Linie verbundene, Menge an Punkten. Optional kann die daraus entstehende Fläche mit einer Farbe gefüllt werden.\\
Um die gewünschte Form zu erzielen, werden 2 Polygone gezeichnet. Das eine (blau) umschließt das andere (rot) und bilden zusammen einen vollen Kreis.\\
\\
Die Nadel wird genau wie der Hintergrund über ein Polygon gezeichnet. Dabei wird die Nadel selbst auf dem Canvas stets richtung Osten gezeichnet. Damit die Nadel schließlich auf den entsprechenden Winkel zeigt, der den aktuellen Wert entspricht, wird nicht die Nadel schräg gezeichnet, sondern stattdessen das Canvas selbst über eine Rotations-Transformation gedreht.


\chapterauthor{Bernd Krupinski}
\subsubsection{LineChart Control Implementierung}
Das LineChart Control wurde ähnlich die das Gauge Control ebenfalls mit 2 Canvas in einer Custom Control in WPF geschrieben. 1 Canvas für den Hintergrund und Skala und 1 Canvas für den Graphen selbst.\\
Das Control ist wie das GaugeControl ebenfalls stark anpassbar. Diese Werte sind unter anderen vertikale Linien zeichnen, horizontale Linien zeichnen, Anzahl horizontale/vertikale Linien, Füllfarbe, Strichfarbe und mehr funktionale Variablen wie Mindestwert, Maximalwert, Fenstergröße, Fensterposition.\\
Für die genaue Erklärungen für jeden Wert siehe in-Code Dokumentation.
Das Control nimmt eine Menge an Punkten an und versucht diese so gut wie möglich darzustellen. Dabei kann das Control nur ein Fenster (Siehe Variablen Fenstergröße, Fensterposition) oder die gesamte Menge anzeigen.\\
Dies geschieht wieder über ein Polygon. Allerdings muss darauf geachtet werden, dass das Canvas eine variable Größe hat. D.h. der Graph muss relativ gezeichnet werden. Es entstehen zwei Fälle:\\
\begin{itemize}
	\item Es existieren mehr Pixel als Sample
	\item Es existieren mehr Sample (Punkte) als Pixel
\end{itemize}
Diese werden im Code unterschieden.\\
\newpage
Für den Fall mehr Pixel als Sample, wird durch jedes Sample iteriert und berechnet wie viele Pixel das momentane Sample entspricht. In der Abbildung (\ref{fig:diagram2}) entspricht jedes Sample mit Abstand von drei, zwei Pixel.\\
\begin{figure}[ht]
	\centering
	\includegraphics[width=.5\textwidth]{TooFewSamples02}
	\caption{Beispiel: Mehr Pixel als Sample}
	\label{fig:diagram2}
\end{figure}

Für den Fall mehr Sample als Pixel, wird wiederum durch jeden Pixel iteriert und berechnet welche Sample für das momentane Pixel relevant sind. Dabei werden stets 2 Pixel gleichzeitig betrachtet, da für die Darstellung das lokale Minimum und Maximum gezeichnet wird.\\

\begin{figure}[ht]
	\centering
	\includegraphics[width=0.4\textwidth]{TooManySamples030001}
	\caption{Beispiel: Mehr Pixel als Sample 01}
	\label{fig:diagram3}
\end{figure}
In dieser Abbildung (\ref{fig:diagram3}) zum Beispiel sind für die rechten zwei Pixel Spalten sechs Sample relevant. Drei rote Sample mit mittel hohen Wert, ein grünes Sample mit einem sehr hohen Wert und zwei blaue Sample mit niedrigen Werten.\\
Das Minimum und Maximum wird bestimmt. Das grüne Sample entspricht dem Maximum, während eins der blauen Sample das Mimimum entspricht. Da das Maximum (grün) rechts vom Minimum (blau) ist, wird entsprechend die linke Spalte das Maximum und die rechte Spalte das Minimum anzeigen.\\
\newpage
\chapterauthor{Ricardo Krause}
\subsection{Integration Controls}
Diese Steuerelemente bieten dem Entwickler diverse Einstellmöglichkeiten in Form und Farbgebung und sind beliebig erweiterbar für kommende Anforderungen. Die beiden Benutzer definierten Steuerelemente wurden zu einem eigenen Control zusammengefasst und mit weiteren Steuerelementen ergänzt. In Abbildung \ref{fig:control1} ist exemplarisch ein komplettes Anzeigeelement für einen Sensorwert abgebildet.

\begin{figure}[ht]
	\centering
		\includegraphics[width=\textwidth]{GUIScreenshot3}
		\caption{Datenanzeige Control}
		\label{fig:control1}
\end{figure}

Die Abbildung \ref{fig:guifinal} zeigt fertige die GUI zum Ende des Projektes. Sie ist aktuell in der Lage die definierten Sensorwerte anzuzeigen, die einzelnen Anzeigeelemente \ref{fig:control1} bieten die Möglichkeiten im Liniendiagramm zu Zoomen und das Wertefenster der X-Achse festzulegen und zu verschieben. Für einstellbare Größen, wie zum Beispiel die Drehgeschwindigkeit kann ein Zielwert eingegeben werden, welcher durch eine zweite rote Nadel im Tachometer angezeigt wird. Im oberen Bereich der GUI kann man die Regelparameter ebenfalls verändern und an den Controller senden.

\begin{figure}[h]
	\centering
		\includegraphics[width=.8\textwidth]{GUIScreenshot2}
		\caption{Ansicht der GUI zum Projektende}
		\label{fig:guifinal}
\end{figure}

Zum Abschluss der Implementierungsphase wurden Unit-Tests geschrieben, wodurch noch einige Fehler aus der Codebasis aufgespürt werden konnten. Diese Tests helfen ebenfalls zukünftigen Entwickler an dem Projekt, da unbedachte Änderungen welche zu Fehlverhalten der Applikation führen können schnell aufgedeckt werden.

\section{Ausblick}
Dieser Abschnitt beinhaltet mögliche Erweiterungen und nicht vollständig umgesetzte Projektteile, welche eine weitere Gruppe als Arbeitspaket aufnehmen kann. 

\begin{description}
\item[Multi-Plattformfähigkeit] Durch die Separierung der einzelnen Codeteile in eigenständige Projekte und die strikte Umsetzung eines Domain Driven Designs kann man die WPF GUI austauschbar machen und durch zum Beispiel eine Web-API ersetzen, welche anschließend mit einem Web-Frontend z.B. ASP.NET oder Angular2 kommuniziert. Auch eine Smartphone Applikation mit zum Beispiel Xamarin wäre möglich.
\item[MouseHover in LineChart] Diese Funktionalität konnte leider nicht Fertiggestellt werden, es wird noch nicht der aktuelle Wert unter der Maus angezeigt.
\item[Änderung des Layoutes] Durch eine Änderung des Layoutes wäre es möglich alle Sensorwerte auf einen Bildschirm Darzustellen, ohne die Notwendigkeit von Scrolbars zu haben. Empfehlung dafür wäre ein Wrappanel mit Horizontaler Ausrichtung.
\item[Hall Pattern Anzeige] Die Aktuelle Anzeige des Hall Pattern ist in einem extra Control unter dem ItemsTemplate für die Anzeigeelemente, dies verhindert Aktuell die einfache Umsetzung der vorher beschriebenen Änderung des Layouts. Hier müsste das Hall Pattern mit in das ItemsTemplate implementiert werden.
\item[ComPort Auswahl] Eine wichtige Erweiterung wäre eine Einstellmöglichkeit für den ComPort, da dieser aktuell Hardcoded ist und ggf. im Quelltext angepasst werden muss. Eine Notlösung wäre eine Auslagerung der Einstellung in eine Konfigurations-Datei. Aber die bessere Lösung wäre eine ComboBox in der GUI zur Auswahl aus verfügbaren Ports. 
\item[Mehrere Inputs] Die GUI könnte noch für mehrere Inputs erweitert werden um es zu ermöglichen mehrere Messstationen oder die Simulation nebeneinander Vergleichend laufen zu lassen.
\item[Benutzer Handbuch] Dem knappen Zeitrahmen geschuldet wurde noch kein Benutzerhandbuch erstellt. Dies wäre Sinnvoll nach der Änderung des Layouts zu erstellen, um aktuelle Bilder verwenden zu können.
\item[Weitere Tests] Sinnvoll für mehr Stabilität wären weitere Tests, einmal für die CustomControls (Gauge und LineChart) Unittests, für die Oberfläche allgemein Coded-UI Tests und ein automatisierter Integrationstest für das Kommunikationsmodul.
\item[Import/Export] Eine Im-/Exportfunktion um Testläufe zu sichern und wiederholt zu laden.
\item[RingSpeicher] Implementierung eines Ringspeichers, welcher eine Maximale Anzahl an Elementen fasst und überschüssige Daten auf der Festplatte oder ähnlich Speichert, um ein Überlaufen des Arbeitsspeichers zu verhindern.
\end{description}

