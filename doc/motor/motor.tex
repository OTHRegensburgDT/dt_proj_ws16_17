\chapterauthor{Andreas Kölbl}

\chapter{Arbeitspaket Motor}
Die Analyse, Implementierung, Test und Integration, sowie die Verkabelung der H-Brücke mit Unterst\"utzung von Herrn Brunner f\"ur Hardware-relevante Problemstellungen war Aufgabe von Herrn Kölbl. Die Drehzahl des Motors soll anschließend gesteuert werden können.

\section{Analysephase}
Aus der Projektspezifikation ergibt sich, dass sich der Motor in zwei Richtungen drehen soll. Auch soll die Drehzahl des Motors einstellbar sein.

\subsection{H-Br\"ucke} ~2Seiten
- Bild relevanter Ausschnitt aus Schaltplan H-Brücke
- Bild von Widerständen wurden ausgelassen für Sternschaltung
- Marker auf relevante Pins
- Erklärung der Kabel mit Bild
\subsection{TI InstaSPIN} ~1 Seite
- GUI mit Oberfläche
- Erklärung, was im Hintergrund läuft
\subsection{POSIF} ~4 Seiten
- Bild von Präsi
- Komplette erklärung des Multi-Channel-Modes
- Quellensuche und fund des XMC4500er Projekts

\section{Ausblick}
Der Code zur Steuerung und Regelung des Motors befindet sich momentan noch in den Anfängen. Bei der aktuellen Bearbeitungslage ist eine vollständige Fertigstellung nicht absehbar. \\
Basierend auf der Ideensammlung aus der Analysephase wäre eine mögliche Erweiterung dieses Arbeitspaketes die Portierung des Codes auf eine Hardwareplattform ohne POSIF-Interface Das hätte eine Neuimplementierung zur Folge. \\
Das Beispiel von dem Chip "XMC44xx"\footnote{\url{https://TODO.html}} zeigt eine dritte Möglichkeit ohne HALL und Quad. Was für Vor/Nachteile wenn
