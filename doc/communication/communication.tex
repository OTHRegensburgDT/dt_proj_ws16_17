\chapter{Arbeitspaket Kommunikation}
Da die Projektspezifikation einen Datenaustausch zwischen $\mu$-Controller und PC enthält, muss diese auch im Projekt realisiert werden. Dabei können auch Parallelen zum ISO/OSI-Schichtenmodell für Netzwerkkommunikation gezogen werden.
\section{Analysephase}
\subsection{Hardware}
Wie auch in der Physical-Layer behandelt, muss die Kommunikation in diesem Projekt zunächst physikalisch erfolgen. 
Dabei stellt sich die Frage, welche Möglichkeiten der Controller zur Datenübertragung an ein externes Gerät bietet. In dessen Handbuch ist aufgeführt, dass dort ein Universal Serial Interface Channel (USIC) Baustein zur Verfügung steht. Dies würde eine serielle Kommunikationsverbindung über UART ermöglichen. Weitere Möglichkeiten sind auf dem Evaluation Board kaum gegeben. Dieses ist zwar für Ethernet vorbereitet, jedoch müssen dafür ein zusätzlicher Chip und eine RJ45-Buchse nachgerüstet werden. Eine selbst definierte und implementierte parallele Schnittstelle mit den GPIO-Ports wäre theoretisch auch möglich, jedoch in der Umsetzung zu komplex und zeitintensiv.
\paragraph{}
Die gemachten Überlegungen haben zur Folge, dass die Kommunikation zwischen Testplatz und PC über die serielle Schnittstelle abgewickelt wird. Nun muss entschieden werden, welches Gerät am Computer verwendet werden soll. Möglich wäre eine Kommunikation über ein serielles Kabel, welches an eine EIA-232 Schnittstelle des PC's angeschlossen wird. Dies ist jedoch mit einigen Problemen behaftet. Zunächst verfügen moderne Desktop Computer nur noch in wenigen Fällen über eine dedizierte EIA-232 Schnittstelle. Dies kann jedoch durch Verwendung eines USB-zu-Seriell-Adapters umgangen werden. Am $\mu$-Controller erfolgt dann der Anschluss an die GPIO-Pins. Allerdings muss dafür ein Adapter angefertigt werden, welcher auf der einen Seite einen D-Sub 9 Stecker und auf der anderen Seite mindestens die Pins 2 (Data Transceive), 3 (Data Receive) und 5 (Ground) als Stifte, passend zur Buchsenleiste des Evaulation Boards, weiterführt. Allerdings werden die Daten an der EIA-232 Schnittstelle mit bis zu 9 Volt übertragen. Dies stellt ein weiteres Problem dar, da an den Pins des Controllers nur 3,3 Volt ausgegeben werden können. Auch kann ein Eingangssignal mit 9 Volt zu Schäden am $\mu$-Controller führen. Um dieses Problem zu umgehen, wäre es möglich, einen USB-zu-Seriell-Adapter mit integriertem Wandler-Chip zu verwenden.
\paragraph{}
Eine weitere Möglichkeit zur seriellen Kommunikation zeigt ein Beispielprojekt von Infineon für das Evaluation Board auf. In diesem erfolgt die Kommunikation zwischen PC und Controller über das USB-Kabel, welches am Debugging-Chip des Boards angeschlossen wird. Auch in diesem Projekt erfolgt der Datenaustausch mit Nutzung des USIC-Bausteins. Allerdings werden hier die Sende- und Empfangsleitungen an den ... Chip weitergeleitet. Für den PC ist darüber hinaus ein Windows-Treiber verfügbar, welcher einen "virtuellen Com-Port" zur Kommunikation mit dem Evaluation Board einrichtet und über den nun Daten im Rahmen des Beispielprojektes ausgetauscht werden können. Auf die Software des PC's hat diese Vorgehensweise keinen Einfluss, da ein virtueller Com-Port in der gleichen Weise verwendet wird, wie ein Hardware-Gerät. Ein weiterer Vorteil dieser Lösung ist, dass eine zusätzliche Stromversorgung des $\mu$-Controllers entfällt, da dies über das USB-Kabel der Kommunikation erfolgt.
\paragraph{}
Die Aufgaben der zweiten Schicht des ISO/OSI-Modells umfassen unter Anderem auch die Sicherung der Datenintegrität während der Übertragung. Dies geschieht im Fall des Ethernet-Protokolls mittels eines Cyclic Redundancy Check Verfahrens. Auch hier bietet der $\mu$-controller mit der Flexible CRC Engine (FCE) einen Hardwarebaustein um Prüfbits für die Kommunikation zu erstellen.
\subsection{Software}

\section{Entwurfsphase}
\section{Implementierung}
\section{Integration}
\section{Ausblick}
