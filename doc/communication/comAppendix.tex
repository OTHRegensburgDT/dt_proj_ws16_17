\chapter{Codeausschnitt aus der C\#-Bibliothek mit ComPort-Settings}
\label{ComAPP:ComPortSettings}
\begin{lstlisting}
private void port_Init()
{
  try
  {
    _uart.BaudRate = 9600;
    _uart.DataBits = 8;
    _uart.StopBits = StopBits.One;
    _uart.Handshake = Handshake.None;
    _uart.Parity = Parity.Even;
    _uart.DtrEnable = false;
    _uart.RtsEnable = false;
    _uart.DataReceived += uart_DataReceived;
    _uart.ReceivedBytesThreshold = 10;
    _uart.Open();
    _isInit = true;
  }
  catch(Exception e)
  {
    throw new SystemException("Uart init failed! Message: " + e.Message);
  }
}
\end{lstlisting}
\chapter{Codeausschnitt aus der C\#-Bibliothek zum Senden der Bytes}
\label{ComAPP:SendWait}
\begin{lstlisting}
  buf = Frameparser.EncapsuleFrame(buf);
  try
  {
    // ReSharper disable once UnusedVariable
    foreach (var elem in buf)
    {
      _uart.Write(buf, i, 1);
      i++;
      System.Threading.Thread.Sleep(30);
    }
    return true;
  }
  catch (Exception e)
  {
    throw new SystemException("Could not send Params! Error: " + e.Message);
  }
\end{lstlisting}

\chapter{Codeausschnitt aus der C\#-Bibliothek mit dem SoF-Automaten}
\label{ComAPP:SoFMachine}
\begin{lstlisting}
//Start of Frame checker
var state = 0; //state 0: no sof; state 1: 0x55 detected; state 2: 0x55 and 0xD5 detected
do{
  spL.Read(SofDel, 0, 1);
  if (state == 0 && SofDel[0] == 0x55)
  {
    state = 1;
    continue;
  }
  if(state == 1 && SofDel[0] == 0x55)
  {
    state = 1;
    continue;
  }
  if (state == 1 && SofDel[0] == 0xD5)
  {
    state = 2;
    continue;
  }
  if (SofDel[0] != 0x55 || SofDel[0] != 0xD5)
  {
    state = 0;
    continue;
  }
}while(state != 2);
state = 0;
//End of Frame Checker
\end{lstlisting}
