\chapter{Projektstart}

Dieses Kapitel beschreibt die Startphase des Projektes. Es werden alle ausgeführten Tätigkeiten, zu beginn des Projektes, aufgezählt und erläutert.

\subsection*{Projektauftrag}

Das Grundlegende Dokument für jedes Projekt ist der Projektauftrag, dieser Auftrag beinhaltet alle relevanten Parameter und Eckdaten des Projektes.

\begin{description}
\item[Projekttitel:] Der Arbeitstitel des Projektes.
\item[Projektnummer:] Eine in der Regel einzigartige Nummer zur Identifikation eines Projektes.
\item[Projektart:] Unterscheidung von verschiedenen möglichen Projekttypen.
\item[Projektleiter:] Bindeglied zwischen Kunden und Entwicklern mit überwachender und steuernder Tätigkeit.
\item[Projektauftraggeber:] Der Auftraggeber des Projektes.
\item[Projektkunden:] Können von Auftraggeber abweichen und auch mehrere sein.
\item[Projektdauer:] Legt den zeitlichen Rahmen des Projektes fest.
\item[Ausgangssituation/Problembeschreibung:] Dies beschreibt den aktuellen Zustand und das Problem, welches mit dem Projekt gelöst werden soll.
\item[Projektgesamtziel:] Zusammengefasste Beschreibung des Projektes mit dem zu erreichenden Ziel.
\item[Projektteilziele und -ergebnisse:] Detaillierte Auflistung der einzelnen Teilziele, mit messbaren zu erreichenden Ergebnissen.
\item[Nicht-Ziele / Nicht-Inhalte:] Abgrenzung des Projektes.
\item[Meilensteine:] Eckpunkte des Projektes mit ziel Datum, welche erreicht werden müssen um das Projekt erfolgreich zum Abschluss bringen zu können.
\item[Randbedingungen und -projektkontext:] Nebenbedingungen welche meist nicht direkt von Projektteam beeinflussbar, aber Notwendig zum erreichen des Projektzieles sind.
\item[Projektklassifizierung:] Parameter, welche das Projekt als Ziffer beschreiben und somit eine Vergleichbarkeit zwischen anderen Projekten schaffen.
\item[Projektorganisation:] Beinhaltet das Projektteam, weitere beteiligte Personen.
\item[Projektressourcen:] Alle für das Projekt zur Verfügung stehenden Ressourcen, beinhalten Personal und Material.
\item[Projektbudget:] Das zur Verfügung stehende Budget.
\item[Wirtschaftlicher oder sonstiger Nutzen:] Dieser Parameter ist meist wichtig für die Entscheider eines Projektes und soll den Gewinn durch das Projekt aufzeigen.
\item[Projektrisiken und -unsicherheiten:] Zeigt alle Risiken auf, welche zu einen Scheitern des Projekt führen könnten.
\item[Projektentscheidung:] Das Projekt muss vor beginn von einer Entscheidungsbefugten Person freigeben werden.
\item[Sonstige relevante Informationen:] Weitere wichtige Information.
\item[Anlagen:] Weitere Dokumente, welche das Projekt betreffen.
\end{description}

Der Projektantrag wurde von Herr Krause aufgesetzt, die Inhalte wurden gemeinsam erarbeitet und mit den Projektbeteiligten Personen abgestimmt. Der ausgearbeitete Antrag befindet sich im Anhang
